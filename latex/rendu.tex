%Classic article
\documentclass[12pt]{article}
%Use the package "geometry" to define the page size
\usepackage[margin=1in]{geometry}

%package for aligning equations
\usepackage{amsmath}

\title{Ray-Cylinder Intersection}
\author{Jules PERRIN, Antonio JIMENEZ, Theo Abel}

\begin{document}
\maketitle

We have a ray defined by his direction and his origin as such:
\begin{align*}
    x(t) = x_0 + u_0 t \\
    y(t) = y_0 + u_1 t \\
    z(t) = z_0 + u_2 t
\end{align*}

And the equation of a cylinder of radius $r$:
\begin{align*}
    y^2 + z^2 = r^2
\end{align*}

So we can just replace $y$ and $z$ in the cylinder equation by the ray equation:
\begin{align*}
    (y_0 + u_1 t)^2 + (z_0 + u_2 t)^2 = r^2
\end{align*}

And we get:
\begin{align*}
    u_1^2 t^2 + 2 u_1 y_0 t + y_0^2 + u_2^2 t^2 + 2 u_2 z_0 t + z_0^2 - r^2 = 0
\end{align*}

By grouping the terms we get:
\begin{align*}
    (u_1^2 + u_2^2)t^2 + (2 u_1 y_0 + 2 u_2 z_0)t + (y_0^2 + z_0^2 - r^2) = 0
\end{align*}

from here we can derive 3 parameters $a$, $b$ and $c$:
\begin{align*}
    a &= u_1^2 + u_2^2 \\
    b &= 2 u_1 y_0 + 2 u_2 z_0 \\
    c &= y_0^2 + z_0^2 - r^2
\end{align*}

So that we have the quadratic equation:
\begin{align*}
    at^2 + bt + c = 0
\end{align*}

That we can easily solve to get the intersection points:
\begin{align*}
    t &= \frac{-b \pm \sqrt{b^2 - 4ac}}{2a}
\end{align*}

They are then 3 differents cases:
\begin{enumerate}
    \item $b^2 - 4ac < 0$: no intersection
    \item $b^2 - 4ac = 0$: one intersection
    \item $b^2 - 4ac > 0$: two intersections
\end{enumerate}

If we have two intersections, we can choose the one that is the closest to the origin of the ray.

One last remark, here we just check for a cylinder of infinite height, but we can easily add a check for the height of the cylinder
after computing the intersection points and just keep the ones that are actually in the cylinder.

\end{document}